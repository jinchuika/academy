\documentclass[11pt]{article}
\usepackage{amsmath,amssymb,amsthm}
\usepackage{graphicx}
\usepackage[margin=1in]{geometry}
\usepackage{fancyhdr}
\usepackage[utf8]{inputenc}
\setlength{\parindent}{0pt}
\setlength{\parskip}{5pt plus 1pt}
\setlength{\headheight}{13.6pt}

\newcommand{\hmwkTitulo}{Tarea\ \#2}
\newcommand{\hmwkFechaEntrega}{01 de marzo de 2015}
\newcommand{\hmwkCurso}{Investigación de Operaciones}
\newcommand{\hmwkCursoSeccion}{Sección A}
\newcommand{\hmwkCursoInstructor}{Ing. Juan Pérez}
\newcommand{\hmwkNombreAutor}{Luis Carlos Contreras}
\newcommand{\hmwkAuthorID}{1990-11-3887}

\newcommand{\hmwkUniversidad}{Universidad Mariano Gálvez de Guatemala}
\newcommand{\hmwkFacultad}{Ingeniería en Sistemas de Información}
\newcommand{\hmwkSede}{Sede Chimaltenango}

\author{\textbf{\hmwkNombreAutor}\\ \textbf{ \vspace{0.1in}\hmwkAuthorID}}
\date{\hmwkFechaEntrega}

\pagestyle{fancy}
\lhead{\textbf{\hmwkNombreAutor}}
\chead{\textbf}
\rhead{\hmwkCurso}

\title{
	\textbf{\hmwkUniversidad} \\
	\hmwkFacultad \\
	\vspace{0.1in}\hmwkSede\\
    \vspace{2.5in}
    \textmd{\textbf{\hmwkTitulo}}\\
    \vspace{0.1in}\large{\hmwkCurso , \textit{\hmwkCursoInstructor }}
    \vspace{2.5in}
}

%Comandos para los problemas
\newcommand\problema[2]{\vspace{.12in}\hrule\textbf{#1: #2}\vspace{.5em}\hrule\vspace{.10in}}
\renewcommand\part[1]{\vspace{.10in}\textbf{(#1)}}
\newcommand\planteamiento{\vspace{.10in}\textbf{Planteamiento: }}
\newcommand\solucion{\vspace{.10in}\textbf{Solución: }}
\newcommand\conclusion{\vspace{.10in}\textbf{Conclusión: }}

\newcommand\funcObj{\vspace{.10in}\textit{Función objetivo: }}

\begin{document}
\maketitle
\pagebreak

\problema{1}{Problema 1}

\planteamiento Una empresa dedicada a la venta de dos productos manufactureros, necesita saber cuántas unidades de cada producto debe producir para el próximo mes si el propósito consiste en maximizar los beneficiones de los mismos. Los detalles de producción son los siguientes: Para producir una unidad del producto A, se necesitan dos horas de máquina y tres horas de mano de obra. Cada unidad del producto B necesita tres horas de tiempo de máquina y dos horas de tiempo de mano de obra. El tiempo semanal disponible de máquina es de 250 horas, mientras que de mano de obra se dispone de 180 horas semanales. Si la ganancia que se obtiene por cada producto A, es de Q.35.00 y por cada producto B, de Q.30.00; determine la utilidad máxima.

\funcObj
\solucion
\begin{table}[h]
\centering
\begin{tabular}{lllll}
\hline
                                  & Máquina A              & Máquina B              &     &  \\ \hline
\multicolumn{1}{l|}{Máquina}      & \multicolumn{1}{l|}{2} & \multicolumn{1}{l|}{3} & {$\leq$ 250} &  \\ \cline{2-3}
\multicolumn{1}{l|}{Mano de obra} & \multicolumn{1}{l|}{3} & \multicolumn{1}{l|}{2} & $\leq$ 180 &  \\ \cline{2-3}
\end{tabular}
\end{table}

\conclusion En caso de que el problema pida alguna conclusión.


\problema{3}{Ejercicio 3}
\planteamiento
Cierto fabricante produce sillas y mesas para las que requiere la utilización de dos sexxiones de producción, la sección de montaje y la de pintura. La producción de una silla requiere una hora de trabajo en la sección de montaje y de dos horas en la de pintura. Por su parte, la fabricación de una mesa precisa tres horas en la sección de montaje y de una hora en la de pintura. La sección de montaje sólo puede estar nueve horas diarias en funcionamiento, mientras que la de pintura sólo ocho horas. El beneficio produciendo mesas es el doble que el de sillas. ¿Cuál ha de ser la producción diaria de mesas y sillas para que el beneficio sea máximo?

\problema{5}{Ejercicio 5}
\planteamiento
Una empresa fabricante ingresó al negocio produciendo una calculadora de 12 funciones, la FC-12, que se vende a Q.6.00. Luego agregó una versión mejorada de 18 funciones, la FC-18, que se vende a Q.9.00. Los costos unitarios de producción son, para la FC.12 de Q.2.00 y para la FC18 de Q.3.00. La empresa puede fabricar cualquier combinación de calculadoras siempre y cuando no exceda su capacidad disponible. El tiempo de montaje de cada calculadora FC-12 es de 0.20 horas, y para la FC-28 es de 0.70 horas; disponiendo de 8,000 hotas de trabajo en el mes. El departamento de mercadeo ha establecido que la empresa puede vender hasta 12,000 calculadoras en el mes sin importar la combinación en la cantidad de cada estilo de calculadora, pues ambas tienen aceptación en el mercado. Determine la cantidad de calculadoras de cada estilo que deben producirse a fin de maximizar la utilidad.

\end{document}
